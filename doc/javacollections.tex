\section{$\ast$~Java collections (Java)}

Java has very good out-of-the-box support for all the data structures discussed in the lectures. In this exercise, you will verify their complexity by running large numbers of operations on them. We'll be using the \code{java.util.Collection} interface, which is an interface common to all java collections. The advantage of this is that it supports similar operations through the same interface. For example, adding an object to a set and pushing an object to the back of an array both use the same method \code{add}.
%
\begin{mybox}{Exercises}
    \begin{enumerate}
        \item Look up the documentation of \code{java.util.Collection}. Try to understand what the different methods are supposed to be doing.
        \item The code in \code{java_collections/JavaCollectionsTest.java} tests a collection by passing it (empty) to the method \code{runTest}. This method then reads a large number of strings from a text file, adds them to the collection, and performs various operations on it. Modify the code to pass it an empty \code{ArrayList} (an implementation of a dynamic array). This class is already imported in the file.
        \item Run the code by executing \code{./run} in the \code{java_collections} directory. It should report the runtimes of various operations.
        \item Run it again. What changed? Can you explain why?
        \item Now run it with several different \code{Collection} instances, such as \code{LinkedList}, \code{TreeSet} and \code{HashSet}. Try to understand the differences in runtimes between them!
            \begin{itemize}
                \item Note that you can add print statements in between the tests to make the output easier to understand. In Java, you can print using \code{System.out.println}.
            \end{itemize}
        \item Some of these classes have constructor parameters to tune their initial configuration. Look up which of them do, and try to pass different values. What is the effect on the runtimes?
            \begin{itemize}
                \item If the runtime fluctuates too much to easily see the difference, try repeating the call to \code{readStrings} to increase the test sample size.
            \end{itemize}
    \end{enumerate}
\end{mybox}

